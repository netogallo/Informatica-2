%%%%%%%%%%%%%%%%%%%%%%%%%%%%%%%%%%%%%%%%%
% Programming/Coding Assignment
% LaTeX Template
%
% This template has been downloaded from:
% http://www.latextemplates.com
%
% Original author:
% Ted Pavlic (http://www.tedpavlic.com)
%
% Note:
% The \lipsum[#] commands throughout this template generate dummy text
% to fill the template out. These commands should all be removed when 
% writing assignment content.
%
% This template uses a Perl script as an example snippet of code, most other
% languages are also usable. Configure them in the "CODE INCLUSION 
% CONFIGURATION" section.
%
%%%%%%%%%%%%%%%%%%%%%%%%%%%%%%%%%%%%%%%%%

%----------------------------------------------------------------------------------------
%	PACKAGES AND OTHER DOCUMENT CONFIGURATIONS
%----------------------------------------------------------------------------------------

\documentclass{article}

\usepackage{fancyhdr} % Required for custom headers
\usepackage{lastpage} % Required to determine the last page for the footer
\usepackage{extramarks} % Required for headers and footers
\usepackage[usenames,dvipsnames]{color} % Required for custom colors
\usepackage{graphicx} % Required to insert images
\usepackage{listings} % Required for insertion of code
\usepackage{courier} % Required for the courier font
\usepackage{multirow}
\usepackage{hyperref}


% Margins
\topmargin=-0.45in
\evensidemargin=0in
\oddsidemargin=0in
\textwidth=6.5in
\textheight=9.0in
\headsep=0.25in

\linespread{1.1} % Line spacing

%----------------------------------------------------------------------------------------
%	CODE INCLUSION CONFIGURATION
%----------------------------------------------------------------------------------------

\definecolor{MyDarkGreen}{rgb}{0.0,0.4,0.0} % This is the color used for comments
\lstloadlanguages{c} % Load Perl syntax for listings, for a list of other languages supported see: ftp://ftp.tex.ac.uk/tex-archive/macros/latex/contrib/listings/listings.pdf
\lstset{language=[sharp]c, % Use Perl in this example
        frame=single, % Single frame around code
        basicstyle=\small\ttfamily, % Use small true type font
        keywordstyle=[1]\color{Blue}\bf, % Perl functions bold and blue
        keywordstyle=[2]\color{Purple}, % Perl function arguments purple
        keywordstyle=[3]\color{Blue}\underbar, % Custom functions underlined and blue
        identifierstyle=, % Nothing special about identifiers                                         
        commentstyle=\usefont{T1}{pcr}{m}{sl}\color{MyDarkGreen}\small, % Comments small dark green courier font
        stringstyle=\color{Purple}, % Strings are purple
        showstringspaces=false, % Don't put marks in string spaces
        tabsize=5, % 5 spaces per tab
        %
        % Put standard Perl functions not included in the default language here
        morekeywords={rand},
        %
        % Put Perl function parameters here
        morekeywords=[2]{on, off, interp},
        %
        % Put user defined functions here
        morekeywords=[3]{test},
       	%
        morecomment=[l][\color{Blue}]{...}, % Line continuation (...) like blue comment
        numbers=left, % Line numbers on left
        firstnumber=1, % Line numbers start with line 1
        numberstyle=\tiny\color{Blue}, % Line numbers are blue and small
        stepnumber=5 % Line numbers go in steps of 5
}

\newcommand{\horrule}[1]{\rule{\linewidth}{#1}}

% Creates a new command to include a perl script, the first parameter is the filename of the script (without .pl), the second parameter is the caption
\newcommand{\perlscript}[2]{
\begin{itemize}
\item[]\lstinputlisting[caption=#2,label=#1]{#1.cs}
\end{itemize}
}

\begin{document}

\begin{tabular}{l l}
\multirow{5}{*}{\includegraphics[width=2cm]{../../Recursos/logo.png}} & Universidad del Istmo de Guatemala \\
 & Facultad de Ingenieria \\
 & Ing. en Sistemas \\
 & Informatica 2 \\
 & Prof. Ernesto Rodriguez - \href{mailto:erodriguez@unis.edu.gt}{erodriguez@unis.edu.gt} \\
\end{tabular}
\\\\\\

\begin{center}
        \horrule{0.5pt}
        \huge{Hoja de trabajo \#1} \\
        \large{Fecha de entrega: 25 de Enero, 2018 - 11:59pm} \\
        \horrule{1pt}
\end{center}

\emph{Instrucciones: Realizar cada uno de los ejercicios siguiendo sus respectivas
instrucciones. El trabajo debe ser entregado a traves de Github, en su repositorio del curso, colocado en una carpeta llamada "Hoja de trabajo 1".
Al menos que la pregunta indique diferente, todas las respuestas a preguntas escritas deben presentarse en
un documento formato pdf, el cual haya sido generado mediante Latex. Los ejercicios de programaci\'on deben ser colocados en una carpeta
llamada ``Programas", la cual debe colocarse dentro de la carpeta correspondiente a esta hoja de trabajo.}

% \perlscript{homework_example}{Sample Perl Script With Highlighting}

\section*{Ejercicio \#1 (10\%)}

\begin{enumerate}
        \item {
                Descargar e instalar el framework \emph{.Net Core} desde: 
                \url{https://www.microsoft.com/net/learn/get-started/windows}. Tambi\'en se recomienda utilizar
                \href{https://code.visualstudio.com/}{Visual Studio Code} con la extension de 
                C\# (\url{https://code.visualstudio.com/Docs/languages/csharp}), pero el estudiante tiene la
                libertad de utilizar cualquier editor o ide para programar.
        }
        \item {
                Dentro de la carpeta ``Programas" de este trabajo, crear una
                nueva carpeta llamada ``QueHaceres". Abrir una terminal en esta
                carpeta y crear un nuevo proyecto de consola mediante .Net Core:
                \\\texttt{> dotnet new console}
        }
\end{enumerate}

\section*{Ejercicio \#2: Crear Proyecto (10\%)}

Dentro de la carpeta del proyecto, definir las siguientes clases en un archivo
separado por clase. El archivo debe tener el mismo nombre que la clase:
\begin{itemize}
        \item{QueHacer}
        \item{QueHaceres}
        \item{Persona}
\end{itemize}
Por el momento no es necesario definir ningun atributo o operaci\'on en escritas
clases. Se llevara a cabo en las siguientes secciones.

\section*{Ejercicio \#3: Definir clases (40\%)}

\subsection*{Propiedad enumerada ``Estado''}

Defina una propiedad enumerada (un \texttt{enum}) llamado ``Estado'' y coloquelo
en el archivo correspondiente a la clase ``QueHacer''. Esta propiedad debe tener
dos campos: ``EnProgreso'' y ``Terminado''/

\subsection*{Clase ``QueHacer''}
La clase ``QueHacer'' debe definir al menos 3 propiedades y una operaci\'on que
haya definido en la hoja anterior. Adicionalmente, debe tener una propiedad llamada
``Estado'' de tipo ``Estado'' que puede ser publicamente leida pero solo modificada
privadametne. Adicionalmente, debe tener una operaci\'on llamada ``Completar'', que
le asigna a la propiedad ``Estado'' el valor ``Terminado''. Esta propiedad debe
tener el valor ``EnProgreso'' cuando el ``QueHacer'' es creado.

El metodo constructor del objeto ``QueHacer'' debe recibir los valores necesarios
para inicializar todas las propiedades de una instancia de ``QueHacer''.

\subsection*{Clase ``Persona''}
La clase persona debe definir las siguientes propiedades:
\begin{itemize}
        \item{``Nombre'' de tipo \texttt{string}}
        \item{``Apellido'' de tipo \texttt{string}}
        \item{``Tareas'' de tipo \texttt{QueHaceres}}
\end{itemize}

Adicionalmente debe definir las siguientes operaciones:
\begin{itemize}
        \item{``EstaDisponible'', que no toma ningun argumento y retorna \texttt{true}
                si el numero de ``QueHacer'' con estado ``EnProgreso'' en la lista
                ``QueHaceres'' de la persona es menor a 1. De lo contrario, false.}
        \item{``AgregarQuehacer'', que toma como parametro un ``QueHacer'' y lo agrega a su
                lista de ``QueHaceres''}.
        \item{``CompletarQuehacer'', que no toma ning\'un parametro y coloca el ``QueHacer''
                m\'as antig\"ua (la cual se le haya asignado lo m\'as antes mediante
                su metodo ``AgregarQuehacer''), con estado ``EnProgreso'', en estado
                ``Terminado'', mediante su metodo ``Finalizar''.} 
\end{itemize}

\subsection*{Clase ``QueHaceres''}
Definir la clase ``QueHaceres'' de tal manera que le permita a la clase ``Persona''
manejar sus ``QueHacer'' mediante los metodos descritos anteriormente. {\bf Importante:}
la clase persona {\bf no} debe tener variables de tipo ``QueHacer''. Toda administraci\'on
de los objetos ``QueHacer'' debe ser realizada por la clase ``QueHaceres''.

\section*{Ejercicio \#4: Presentar resultados (20\%)}
El comando \texttt{dotnet new console} crea automaticametne un archivo llamado
``Program.cs''. En este archivo existe un metodo llamado ``main''. Este es el
metodo que ejecutara el comando \texttt{dotnet run}, el punto de entrada al
programa. Este metodo debe realizar lo siguiente:
\begin{enumerate}
        \item{Crear dos instancias de la clase ``Persona'', me referire a ellos
                como \emph{Persona1} y \emph{Persona2}}
        \item{Crear seis instancias de la clase ``QueHacer''}
        \item{Agregar 3 ``QueHacer'' a cada instancia de ``Persona'' mediante
                su metodo ``AgregarQuehacer'' no se debe agregar el mismo
                ``QueHacer'' a dos personas diferentes.}
        \item{Llamar el metodo ``CompletarQuehacer'' una vez con ``Persona1''
                y tres veces con ``Persona2''.}
        \item{Imprimir en la consola mediante el metodo \texttt{Console.WriteLine}
                el resultado del metodo ``EstaDisponible'' de ambas peersonas.
                {\bf Nota:} ``Persona1'' no debe estar disponible mientras que
                ``Persona2'' si, luego de la ejecuci\'on del programa.}
\end{enumerate}

\section*{Ejercicio \#5: Confusion (20\%)}

\perlscript{./Programa}{Confusion}

Luego de ejecutar el programa que se muestra en esta secci\'on, el objeto
``alonzo'' descubre que se encuentra disponible. Sin embargo, el metodo
``CompletarQuehacer'' nunca fue ejecutado en dicho objeto. ¿A que se debe
este extra\~no resultado? Por colocar su respuesta en un documento Latex
y entregarlo como parte de la tarea.

\bibliography{../../Referencias/referencias}
\bibliographystyle{plain}

\end{document}