%%%%%%%%%%%%%%%%%%%%%%%%%%%%%%%%%%%%%%%%%
% Programming/Coding Assignment
% LaTeX Template
%
% This template has been downloaded from:
% http://www.latextemplates.com
%
% Original author:
% Ted Pavlic (http://www.tedpavlic.com)
%
% Note:
% The \lipsum[#] commands throughout this template generate dummy text
% to fill the template out. These commands should all be removed when 
% writing assignment content.
%
% This template uses a Perl script as an example snippet of code, most other
% languages are also usable. Configure them in the "CODE INCLUSION 
% CONFIGURATION" section.
%
%%%%%%%%%%%%%%%%%%%%%%%%%%%%%%%%%%%%%%%%%

%----------------------------------------------------------------------------------------
%	PACKAGES AND OTHER DOCUMENT CONFIGURATIONS
%----------------------------------------------------------------------------------------

\documentclass{article}

\usepackage{fancyhdr} % Required for custom headers
\usepackage{lastpage} % Required to determine the last page for the footer
\usepackage{extramarks} % Required for headers and footers
\usepackage[usenames,dvipsnames]{color} % Required for custom colors
\usepackage{graphicx} % Required to insert images
\usepackage{listings} % Required for insertion of code
\usepackage{courier} % Required for the courier font
\usepackage{multirow}
\usepackage{hyperref}


% Margins
\topmargin=-0.45in
\evensidemargin=0in
\oddsidemargin=0in
\textwidth=6.5in
\textheight=9.0in
\headsep=0.25in

\linespread{1.1} % Line spacing

%----------------------------------------------------------------------------------------
%	CODE INCLUSION CONFIGURATION
%----------------------------------------------------------------------------------------

\definecolor{MyDarkGreen}{rgb}{0.0,0.4,0.0} % This is the color used for comments
\lstloadlanguages{c} % Load Perl syntax for listings, for a list of other languages supported see: ftp://ftp.tex.ac.uk/tex-archive/macros/latex/contrib/listings/listings.pdf
\lstset{language=[sharp]c, % Use Perl in this example
        frame=single, % Single frame around code
        basicstyle=\small\ttfamily, % Use small true type font
        keywordstyle=[1]\color{Blue}\bf, % Perl functions bold and blue
        keywordstyle=[2]\color{Purple}, % Perl function arguments purple
        keywordstyle=[3]\color{Blue}\underbar, % Custom functions underlined and blue
        identifierstyle=, % Nothing special about identifiers                                         
        commentstyle=\usefont{T1}{pcr}{m}{sl}\color{MyDarkGreen}\small, % Comments small dark green courier font
        stringstyle=\color{Purple}, % Strings are purple
        showstringspaces=false, % Don't put marks in string spaces
        tabsize=5, % 5 spaces per tab
        %
        % Put standard Perl functions not included in the default language here
        morekeywords={rand},
        %
        % Put Perl function parameters here
        morekeywords=[2]{on, off, interp},
        %
        % Put user defined functions here
        morekeywords=[3]{test},
       	%
        morecomment=[l][\color{Blue}]{...}, % Line continuation (...) like blue comment
        numbers=left, % Line numbers on left
        firstnumber=1, % Line numbers start with line 1
        numberstyle=\tiny\color{Blue}, % Line numbers are blue and small
        stepnumber=5 % Line numbers go in steps of 5
}

\newcommand{\horrule}[1]{\rule{\linewidth}{#1}}

% Creates a new command to include a perl script, the first parameter is the filename of the script (without .pl), the second parameter is the caption
\newcommand{\perlscript}[2]{
\begin{itemize}
\item[]\lstinputlisting[caption=#2,label=#1]{#1.cs}
\end{itemize}
}

\begin{document}

\begin{tabular}{l l}
\multirow{5}{*}{\includegraphics[width=2cm]{../../Recursos/logo.png}} & Universidad del Istmo de Guatemala \\
 & Facultad de Ingenieria \\
 & Ing. en Sistemas \\
 & Informatica 2 \\
 & Prof. Ernesto Rodriguez - \href{mailto:erodriguez@unis.edu.gt}{erodriguez@unis.edu.gt} \\
\end{tabular}
\\\\\\

\begin{center}
        \horrule{0.5pt}
        \huge{Hoja de trabajo \#5} \\
        \large{Fecha de entrega: 1 de Marzo, 2018 - 11:59pm} \\
        \horrule{1pt}
\end{center}
\emph{Instrucciones: Realizar cada uno de los ejercicios siguiendo sus respectivas
instrucciones. El trabajo debe ser entregado a traves de Github, en su repositorio del curso, colocado en una carpeta llamada ``Hoja de trabajo 5''.
Al menos que la pregunta indique diferente, todas las respuestas a preguntas escritas deben presentarse en
un documento formato pdf, el cual haya sido generado mediante Latex. Los ejercicios de programaci\'on deben ser colocados en una carpeta
llamada ``Programas", la cual debe colocarse dentro de la carpeta correspondiente a esta hoja de trabajo.}

% \perlscript{homework_example}{Sample Perl Script With Highlighting}

\section*{Reduce}
La funci\'on de orden superior m\'as utilizada probablemente es la funci\'on \emph{Reduce}
(tambien conocida como \emph{fold}). En esta tarea se implementara
dicha funci\'on y se llevara a cabo paso a paso.
\\\\
Para empezar crear una soluci\'on dentro de la carpeta ``Programas''
y dentro de ella crear dos proyectos: ``Reduce'' y ``ReduceTests''.
El proyecto ``Reduce'' debe ser de tipo \emph{console} y ``ReduceTests''
de tipo \emph{xunit}.

\section*{Parte 1 (50\%)}
En la clase ``Program'' del archivo ``Program.cs'', definir un
metodo estatico llamado ``ReduceInt''. Este metodo tiene tipo
``$\mathtt{ReduceInt}\ :\ \mathtt{int[]}\otimes\mathtt{int}\otimes
(\mathtt{int}\otimes\mathtt{int}\rightarrow\mathtt{int})\rightarrow\mathtt{int}$'',
en otras palabras:
\begin{itemize}
        \item{Acepta un arreglo de \texttt{int} como primer parametro}
        \item{Acepta un \texttt{int} como segundo parametro}
        \item{Acepta una funci\'on que toma dos \texttt{int} y produce un \texttt{int}
        como tercer parametro.}
        \item{Retorna un \texttt{int}}
\end{itemize}
La funci\'on opera de la siguiente manera:
\begin{enumerate}
        \item{Inicializa una variable de tipo \texttt{int} con el
        valor del segundo parametro. Esta variable se llama \emph{acumulador}}
        \item{Luego un ciclo recorre todos los elementos.}
        \item{En cada iteraci\'on del ciclo, el \emph{acumulador} y el
        elemento actual se utilizan para llamar la funci\'on en el tercer
        parametro (conocida como la \emph{reducci\'on}), y el resultado de
        dicha funci\'on se vuelbe el nuevo valor del \emph{acumulador}}
        \item{Luego de recorrer todos los elementos, el resultado de
        reduce es el ultimo valor del \emph{acumulador}}
\end{enumerate}
Ejemplos:
\perlscript{./ReduceInt}{}
Como primer ejercicio, implementar la funcion ``ReduceInt'' e
implementar 1 prueba unitaria (en el proyecto ``ReduceTests'')
que verifique el comportameinto correcto de ``ReduceInt''.
\pagebreak
\section*{Parte 2 (50\%)}
La forma m\'as general de la funci\'on \emph{Reduce}, es una version
con dos parametros genericos, los llamaremos \texttt{T} y \texttt{A}.
El primer parametro corresponde al tipo del arreglo que recibe la
funci\'on \emph{Reduce}, el segundo corresponde al tipo del \emph{accumulador}.
Esto conlleva a que la funci\'on tenga el tipo ``$\mathtt{Reduce}\langle\mathtt{T},
\mathtt{A}\rangle\ :\ \mathtt{T}[]\otimes\mathtt{A}\otimes(\mathtt{A}\otimes\mathtt{T}
\rightarrow\mathtt{A})\rightarrow\mathtt{A}$''. Esta version puede
trabajar con arreglos y acumuladores de tipo diferentes, dando la
oportunidad de implementar una variedad de funciones. Por ejemplo:
\perlscript{./Reduce}{}
Como segundo ejercicio debe implementar la funci\'on ``Reduce'' e
implementar una prueba unitaria (en el proyecto ``ReduceTests'')
que verifique el comportamiento de ``Reduce''.
\end{document}