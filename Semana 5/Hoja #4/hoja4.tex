%%%%%%%%%%%%%%%%%%%%%%%%%%%%%%%%%%%%%%%%%
% Programming/Coding Assignment
% LaTeX Template
%
% This template has been downloaded from:
% http://www.latextemplates.com
%
% Original author:
% Ted Pavlic (http://www.tedpavlic.com)
%
% Note:
% The \lipsum[#] commands throughout this template generate dummy text
% to fill the template out. These commands should all be removed when 
% writing assignment content.
%
% This template uses a Perl script as an example snippet of code, most other
% languages are also usable. Configure them in the "CODE INCLUSION 
% CONFIGURATION" section.
%
%%%%%%%%%%%%%%%%%%%%%%%%%%%%%%%%%%%%%%%%%

%----------------------------------------------------------------------------------------
%	PACKAGES AND OTHER DOCUMENT CONFIGURATIONS
%----------------------------------------------------------------------------------------

\documentclass{article}

\usepackage{fancyhdr} % Required for custom headers
\usepackage{lastpage} % Required to determine the last page for the footer
\usepackage{extramarks} % Required for headers and footers
\usepackage[usenames,dvipsnames]{color} % Required for custom colors
\usepackage{graphicx} % Required to insert images
\usepackage{listings} % Required for insertion of code
\usepackage{courier} % Required for the courier font
\usepackage{multirow}
\usepackage{hyperref}


% Margins
\topmargin=-0.45in
\evensidemargin=0in
\oddsidemargin=0in
\textwidth=6.5in
\textheight=9.0in
\headsep=0.25in

\linespread{1.1} % Line spacing

%----------------------------------------------------------------------------------------
%	CODE INCLUSION CONFIGURATION
%----------------------------------------------------------------------------------------

\definecolor{MyDarkGreen}{rgb}{0.0,0.4,0.0} % This is the color used for comments
\lstloadlanguages{c} % Load Perl syntax for listings, for a list of other languages supported see: ftp://ftp.tex.ac.uk/tex-archive/macros/latex/contrib/listings/listings.pdf
\lstset{language=[sharp]c, % Use Perl in this example
        frame=single, % Single frame around code
        basicstyle=\small\ttfamily, % Use small true type font
        keywordstyle=[1]\color{Blue}\bf, % Perl functions bold and blue
        keywordstyle=[2]\color{Purple}, % Perl function arguments purple
        keywordstyle=[3]\color{Blue}\underbar, % Custom functions underlined and blue
        identifierstyle=, % Nothing special about identifiers                                         
        commentstyle=\usefont{T1}{pcr}{m}{sl}\color{MyDarkGreen}\small, % Comments small dark green courier font
        stringstyle=\color{Purple}, % Strings are purple
        showstringspaces=false, % Don't put marks in string spaces
        tabsize=5, % 5 spaces per tab
        %
        % Put standard Perl functions not included in the default language here
        morekeywords={rand},
        %
        % Put Perl function parameters here
        morekeywords=[2]{on, off, interp},
        %
        % Put user defined functions here
        morekeywords=[3]{test},
       	%
        morecomment=[l][\color{Blue}]{...}, % Line continuation (...) like blue comment
        numbers=left, % Line numbers on left
        firstnumber=1, % Line numbers start with line 1
        numberstyle=\tiny\color{Blue}, % Line numbers are blue and small
        stepnumber=5 % Line numbers go in steps of 5
}

\newcommand{\horrule}[1]{\rule{\linewidth}{#1}}

% Creates a new command to include a perl script, the first parameter is the filename of the script (without .pl), the second parameter is the caption
\newcommand{\perlscript}[2]{
\begin{itemize}
\item[]\lstinputlisting[caption=#2,label=#1]{#1.cs}
\end{itemize}
}

\begin{document}

\begin{tabular}{l l}
\multirow{5}{*}{\includegraphics[width=2cm]{../../Recursos/logo.png}} & Universidad del Istmo de Guatemala \\
 & Facultad de Ingenieria \\
 & Ing. en Sistemas \\
 & Informatica 2 \\
 & Prof. Ernesto Rodriguez - \href{mailto:erodriguez@unis.edu.gt}{erodriguez@unis.edu.gt} \\
\end{tabular}
\\\\\\

\begin{center}
        \horrule{0.5pt}
        \huge{Hoja de trabajo \#4} \\
        \large{Fecha de entrega: 22 de Febrero, 2018 - 11:59pm} \\
        \horrule{1pt}
\end{center}
\emph{Instrucciones: Realizar cada uno de los ejercicios siguiendo sus respectivas
instrucciones. El trabajo debe ser entregado a traves de Github, en su repositorio del curso, colocado en una carpeta llamada ``Hoja de trabajo 4''.
Al menos que la pregunta indique diferente, todas las respuestas a preguntas escritas deben presentarse en
un documento formato pdf, el cual haya sido generado mediante Latex. Los ejercicios de programaci\'on deben ser colocados en una carpeta
llamada ``Programas", la cual debe colocarse dentro de la carpeta correspondiente a esta hoja de trabajo.}

% \perlscript{homework_example}{Sample Perl Script With Highlighting}

\section*{Iniciaci\'on}
Crear una soluci\'on para este deber. La soluci\'on debe tener 2 projectos: \emph{Genericos}
y \emph{GenericosTests}. El projecto \emph{Genericos} debe ser de tipo \emph{console} y el
proyecto \emph{GenericosTests} de tipo \emph{xunit}. Finalmente definir una clase llamada
\emph{Genericos} en el proyecto \emph{Genericos} y una clase llamada \emph{GenericosTests}
en el proyecto \emph{GenericosTests}.

\section*{Ejercicio \#1 (20\%)}
Escriba un metodo estatico generico en la clase \emph{Genericos} llamado \emph{Head}, con un parametro generico \texttt{T},
el cual tiene tipo ``$\mathtt{Head}:\ \mathtt{T[\ ]}\ \rightarrow\ \mathtt{T}$''. Este metodo recibe un arreglo de elementos y
retorna el primer elemento del arreglo. {\bf Nota:} No es necesario revisar que el arreglo
tenga al menos un elemento.
\\\\
Escribir una prueba unitaria en la clase \emph{GenericosTests} llamada \emph{TestHead}
que verifique el funcionamiento correcto de este metodo. Esto significa que verifica
que el objeto retornado sea el primer objeto del arreglo dado.

\section*{Ejercio \#2 (20\%)}
Escribir un metodo generico llamado \emph{Tail}, con un parametro generico \texttt{T},
de tipo ``$\mathtt{Tail}:\ \mathtt{T[\ ]}\ \rightarrow\ \mathtt{T[\ ]}$'', que acepta un arreglo
como parametro y retorna un arreglo con todos los elementos, en el mismo orden a excepci\'on
del primer elemento del arreglo original.
\\\\
Escribir una prueba unitaria para este metodo en la clase \emph{GenericosTests}, llamada
\emph{TestTail}, que verifique el funcionamiento correcto de esta funci\'on.

\section*{Ejercicio \#3 (40\%)}
\perlscript{Tupla}{}
Utilizar la clase \emph{Tupla}, que esta definida en esta seccion (debe incluirla
en su codigo), para definir en la clase \emph{Genericos} un metodo generico llamado \emph{Zip}. Este metodo recibe
dos parametros genericos (\texttt{T1} y \texttt{T2}) y tiene tipo ``$\mathtt{Zip}\ 
\mathtt{T1[\ ]}\otimes\mathtt{T2[\ ]}\ \rightarrow\ \mathtt{Tupla\langle T1,T2\rangle[\ ]}$''.
Este metodo recibe como parametro dos arreglos y retorna un arreglo donde el cada elemento
del primer arreglo es emparejado con el elemento del segundo arreglo con el mismo indice.
En caso que los arreglos pasados como parametro sean de longitud diferente, el resultado
debe tener la misma longitud que el arreglo m\'as corto.
\\\\
Escribir una prueba unitaria llanada \emph{TestZip} en la clase \emph{GenericosTests} que
verifique el funcionamiento correcto del metodo \emph{Zip}. Queda a su criterio determinar
como verificar dicho funcionamiento.

\section*{Ejercicio \#4 (20\% + 10\%)}
Escriba un programa ejemplo (no lo coloque en el proyecto, sino que en la misma carpeta
que sus archivos latex de esta tarea) que muestre las ventajas que tiene el metodo \emph{Head}
al utilizar tipos genericos sobre una version equivalente de ese metodo que en vez de
quenericos, reciba un arreglo de tipo \texttt{object} y retorne un \texttt{object}. Brevemetne,
explicar dicho programa en un archivo latex. Se le otorgara un 10\% extra en esta tarea si
incluye su codigo fuente en el pdf final (asi como en este pdf).

\end{document}