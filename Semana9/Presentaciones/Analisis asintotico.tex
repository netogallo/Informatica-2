\documentclass{beamer}
\usepackage[utf8]{inputenc}
\usepackage{hyperref}
\usepackage{multicol}
\usepackage{multirow}
\usepackage{hyperref}
\usepackage{verbatim}
\usepackage{algpseudocode} 

\inputencoding{utf8}

\mode<presentation> {
    \usetheme{Madrid}
}

\usepackage{graphicx}
\usepackage{booktabs}

\title[Git]{Analisis Asintotico}
\author{Ernesto Rodriguez}
\institute{
    Universidad del Itsmo \\
    \medskip \textit{erodriguez@unis.edu.gt}
}

\date[\today]{}

\begin{document}

\begin{frame}
\titlepage
\end{frame}

\begin{frame}
\frametitle{Tiempo de Ejecuci\'on}
\begin{itemize}
    \item{Depende de los parametros. Ej. un arreglo ordenado
    es m\'as facil ordenar que un arreglo desordenado.}
    \item{Depende de la maquina que este ejecutando el programa.}
    \item{Depende del tama\~no de los parametros. Una sequencia
    corta es m\'as facil ordenar que una secuencia larga.}
\end{itemize}
{\bf Idea: } Dar el tiempo de ejecuci\'on con respecto al tama\~no de los parametros para
el peor caso posible.
\end{frame}

\begin{frame}
\frametitle{Tipos de Analisis}
\begin{itemize}
    \item{{\bf Peor caso: }$T(n)$ tiempo maximo de ejecuci\'on para un parametro con tama\~no $n$.}
    \item{{\bf Tiempo promedio: }$T(n)$ tiempo promedio de ejecici\'on para un parametro con tama\~no $n$.
        Requiere asumir que los parametros siguen una distribuci\'on.}
    \item{{\bf Mejor caso: } $T(n)$ el mejor tiempo de ejecici\'on possible para un parametro
        con tama\~no $n$.}
\end{itemize}
\end{frame}

\begin{frame}
\frametitle{Hardware}
\begin{itemize}
    \item{La velocidad de ejecuci\'on de una computadora permanece constante}
    \item{Podemos utilizar este hecho para definir una medida universal}
\end{itemize}
{\bf Idea: } Ignorar las diferencias que permanecen constantes y enfocarse
en la ejecici\'on del algoritmo cuando el tama\~no de los parametros crece al infinito.
\end{frame}

\begin{frame}
\frametitle{Limite asintotico $\Theta(n)$}
Para toda funci\'on asintotica positiva $g(n)$, definimos:
\[
    \Theta(g(n))\ :=\{
        \begin{array}{l c l}
            \multirow{2}{*}{f(n)} & \multirow{2}{*}{:} & \exists c_1, c_2, n_0\ tq. \\
            & & \forall n. n>n_0 \wedge 0 \leq c_1g(n) \leq f(n)\leq c_2(g(n))
        \end{array}
    \}
\]
Ejemplo: $3x^3 + 6x^2\in \Theta(n^3)$
\\
{\bf Nota: } Cuanod $n$ es muy grande, $n^2$ es mejor que $n^3$
\end{frame}

\begin{frame}
\frametitle{Limite superior: Notaci\'on O}
Para toda funci\'on $g(n)$, definimos:
\[
    \mathcal{O}(g(n))\ :=\{
        \begin{array}{l c l}
            \multirow{2}{*}{f(n)} & \multirow{2}{*}{:} & \exists c, n_0\ tq. \\
            & & \forall n. n>n_0 \wedge 0 \leq f(n)\leq c(g(n))
        \end{array}
    \}
\]
{\bf Nota: } Se dice que una funci\'on $f(x)$ esta \emph{delimitada polinomialmente}
si $\exists k.\ k>0\wedge f(x)\in\mathcal{O}(n^k)$
\end{frame}

\begin{frame}
\frametitle{Limite inferior: Notaci\'on $\Omega$}
Para toda funci\'on $g(n)$, definimos:
\[
    \Omega(g(n))\ :=\{
        \begin{array}{l c l}
            \multirow{2}{*}{f(n)} & \multirow{2}{*}{:} & \exists c, n_0\ tq. \\
            & & \forall n. n>n_0 \wedge 0 \leq g(n)\leq c(f(n))
        \end{array}
    \}
\]
{\bf Nota: } $\Theta(g(n))\equiv \mathcal{O}(g(n))\cap \Omega(g(n))$
\end{frame}

\begin{frame}
\frametitle{Analisis de ``InsertionSort''}

\begin{algorithmic}
\Function{Insertion-Sort}{A,n}
\For{j=2 to n}
    \State $key \gets A[j]$
    \State $i \gets j-1$
    \While{$i>0\wedge A[i]>key$}
        \State $A[i+1] \gets A[i]$
        \State $i \gets i-1$
    \EndWhile
    \State $A[i+1] \gets key$
\EndFor
\EndFunction
\end{algorithmic}

\end{frame}

\end{document}