%%%%%%%%%%%%%%%%%%%%%%%%%%%%%%%%%%%%%%%%%
% Programming/Coding Assignment
% LaTeX Template
%
% This template has been downloaded from:
% http://www.latextemplates.com
%
% Original author:
% Ted Pavlic (http://www.tedpavlic.com)
%
% Note:
% The \lipsum[#] commands throughout this template generate dummy text
% to fill the template out. These commands should all be removed when 
% writing assignment content.
%
% This template uses a Perl script as an example snippet of code, most other
% languages are also usable. Configure them in the "CODE INCLUSION 
% CONFIGURATION" section.
%
%%%%%%%%%%%%%%%%%%%%%%%%%%%%%%%%%%%%%%%%%

%----------------------------------------------------------------------------------------
%	PACKAGES AND OTHER DOCUMENT CONFIGURATIONS
%----------------------------------------------------------------------------------------

\documentclass{article}

\usepackage{fancyhdr} % Required for custom headers
\usepackage{lastpage} % Required to determine the last page for the footer
\usepackage{extramarks} % Required for headers and footers
\usepackage[usenames,dvipsnames]{color} % Required for custom colors
\usepackage{graphicx} % Required to insert images
\usepackage{listings} % Required for insertion of code
\usepackage{courier} % Required for the courier font
\usepackage{multirow}
\usepackage{hyperref}


% Margins
\topmargin=-0.45in
\evensidemargin=0in
\oddsidemargin=0in
\textwidth=6.5in
\textheight=9.0in
\headsep=0.25in

\linespread{1.1} % Line spacing

%----------------------------------------------------------------------------------------
%	CODE INCLUSION CONFIGURATION
%----------------------------------------------------------------------------------------

\definecolor{MyDarkGreen}{rgb}{0.0,0.4,0.0} % This is the color used for comments
\lstloadlanguages{c} % Load Perl syntax for listings, for a list of other languages supported see: ftp://ftp.tex.ac.uk/tex-archive/macros/latex/contrib/listings/listings.pdf
\lstset{language=[sharp]c, % Use Perl in this example
        frame=single, % Single frame around code
        basicstyle=\small\ttfamily, % Use small true type font
        keywordstyle=[1]\color{Blue}\bf, % Perl functions bold and blue
        keywordstyle=[2]\color{Purple}, % Perl function arguments purple
        keywordstyle=[3]\color{Blue}\underbar, % Custom functions underlined and blue
        identifierstyle=, % Nothing special about identifiers                                         
        commentstyle=\usefont{T1}{pcr}{m}{sl}\color{MyDarkGreen}\small, % Comments small dark green courier font
        stringstyle=\color{Purple}, % Strings are purple
        showstringspaces=false, % Don't put marks in string spaces
        tabsize=5, % 5 spaces per tab
        %
        % Put standard Perl functions not included in the default language here
        morekeywords={rand},
        %
        % Put Perl function parameters here
        morekeywords=[2]{on, off, interp},
        %
        % Put user defined functions here
        morekeywords=[3]{test},
       	%
        morecomment=[l][\color{Blue}]{...}, % Line continuation (...) like blue comment
        numbers=left, % Line numbers on left
        firstnumber=1, % Line numbers start with line 1
        numberstyle=\tiny\color{Blue}, % Line numbers are blue and small
        stepnumber=5 % Line numbers go in steps of 5
}

\newcommand{\horrule}[1]{\rule{\linewidth}{#1}}

% Creates a new command to include a perl script, the first parameter is the filename of the script (without .pl), the second parameter is the caption
\newcommand{\perlscript}[2]{
\begin{itemize}
\item[]\lstinputlisting[caption=#2,label=#1]{#1.cs}
\end{itemize}
}

\begin{document}

\begin{tabular}{l l}
\multirow{5}{*}{\includegraphics[width=2cm]{../Recursos/logo.png}} & Universidad del Istmo de Guatemala \\
 & Facultad de Ingenieria \\
 & Ing. en Sistemas \\
 & Proyecto Final \\
 & Prof. Ernesto Rodriguez - \href{mailto:erodriguez@unis.edu.gt}{erodriguez@unis.edu.gt} \\
\end{tabular}
\\\\\\

\begin{center}
        \horrule{0.5pt}
        \huge{Proyecto Final} \\
        \large{Fecha de entrega intermedia: 20 de Abril, 2018 - 11:59pm} \\
        \large{Fecha de entrega final: 14 de Mayo, 2018 - 11:59pm} \\
        \horrule{1pt}
\end{center}

El proyecto final tiene como intenci\'on poner en practica lo
que se aprendio en clase. Adicionalmente, tiene el objetivo
motivar al estudiante a trabajar en grupo por lo cual debe
realizarse en grupos de 2 o 3 personas.
\\\\
Este proyecto se enregara en dos etapas: la entrega intermediaria
y la entrega final. Su valor total es un 30\% de la nota del curso
ya que la entrega intermediaria sera en vez del 3er examen parcial
y la entrega final tendra el valor de medio examen final.
\\\\
El proyecto debe ser entregado mediante Github. Se debe crear un
{\bf repositorio exclusivo} para este proyecto (uno por proyecto).
La ubicaci\'on del repositorio debe ser enviada a 
\href{mailto:erodriguez@unis.edu.gt}{erodriguez@unis.edu.gt}.
\\\\
Para este proyecto, los estudiantes deben presentar su propia
version del juego ``Connect 4''. Un ejemplo de dicho juego se
encuentra en \url{http://www.connectfour.org/connect-4-online.php}.
\\\\
En resumen, es un juego de 2 jugadores en el cual los jugadores
toman turnos colocando una moneda en alguna de las columnas.
Esta moneda (por gravedad) se deslizara hasta ocupar el espacio
m\'as bajo posible. El juego concluye cuando alguno de los dos
jugadores ha logrado hacer una fila de 4 monedas.
\\\\
Este documento probablemente no es una descripci\'on completa
del proyecto y se espera que el estudiante (como en la vida real)
le pregunte al profesor que aclare los detalles que no estan
claros.
\section*{Proyectos alternos}
Si un grupo desea hacer otro proyecto, por favor hablar con el
catedratico para llegar a un acuerdo sobre la entrega intermedia y
final.

\section*{Entrega final (57\%)}
\begin{itemize}
        \item{Interfaz utilizando caracteres impresos en la terminal
        (no es requisito una interfaz grafica)}
        \item{Controles del juego mediante el teclado, los alumnos
        tienen la libertad de dise\~nar como van a interactuar
        los jugadores con el juego (no es necesario que funcione con
        el mouse.)}
        \item{Modalidad de juego para dos jugadores {\bf humanos}. No
        es necesario que el juego tenga una inteligencia artifical,
        sino que ambos jugadores tomaran turnos en la misma computadora.}
        \item{El juego debe detectar automaticamente cuando un jugador
        ha ganado y mostrar un mensaje de victora.}
        \item{El juego debe presentar instrucciones claras acerca
        de la forma que los usuarios deben interactuar con el juego.}
        \item{Se evaluara la calidad del codigo, esto significa:
        \begin{itemize}
                \item{Utilizaci\'on correcta de la abstracci\'on.}
                \item{Utilizaci\'on correcta de herencia e interfaces.}
                \item{Principio de unica responsabilidad aplicado a las clases.}
                \item{Utilizaci\'on de \emph{git} para llevar control del codigo.}
                \item{Utilizaci\'on de colecciones y algoritmos para su manipulaci\'on.}
                \item{Pruebas unitarias.}
        \end{itemize}
        \item{Los estudiantes presentaran y explicaran el codigo al catedratico
        y el tiene la libertad de hacer cualquier pregunta a cualquier miembro
        sobre cualquier aspecto del codigo.}
        }
\end{itemize}

\section*{Entrega intermediaria (33\%)}
Este proyecto tendra una entrega intermediaria. La intenci\'on es
motivar al estudiante que empieze a trabajar en el proyecto lo antes
posible y que desarrolle el software de forma \emph{ordenada}. Para
esta entrega deben haber lo siguiente:
\begin{itemize}
        \item{5 interfaces, las cuales definiran el comportamiento
        de los diversos componentes del juego. No es necesario que
        tengan una implementaci\'on en este punto.}
        \item{10 pruebas unitarias dise\~nadas por el estudiante.
        Estas pruebas deben verificar el comportamiento correcto
        de varios de los componentes del juego.}
        \item{El codigo necesario para que 6 de las 10
        pruebas unitarias se ejecuten correctamente.}
\end{itemize}
Las interfaces deben estar correctamente documentadas, indicando
en la documentaci\'on (mediante comentarios) cual es el proposito
de la interfaz y cual es el proposito de cada uno de los metodos
de la interfaz. A pesar que solo 6 de las 10 pruebas unitarias
presentadas en este punto deben ejecutarse correctamente, tanto
el juego como las pruebas deben compilar correctamente. Si es
necesario, se le puede dar una implementaci\'on temporal e
incorrecta a algunos metodos para que ambos compilen (ie.
solo retornar un valor arbitrario por ejemplo.)

\section*{Extras (Hasta un 30\% adicional)}
Puede notarse que 57\% + 33\% suman 90\%. Para sacar nota comleta,
el juego debe implementar algunos extras. Los extras tambien le
permiten obtener puntos adicionales que seran abonados en otras
notas que haya sacado en el curso. Ejemplos de algunos extras son:
\begin{itemize}
        \item{Utilizar un lenguaje de programaci\'on distinto a C\#
        (Haskell, Java, Python, Javascript, Lisp, todo menos Objective-C y Swift)}
        \item{Interfaz grafica (GUI), puede hacerse con WPF, GTK, QT, OpenGL o Web
        (No se permite Cocoa debido a que no tengo una Mac)}
        \item{Version movil (solamente Android debido a que no tengo IPhone)}
        \item{Modo de juego contra una inteligencia artificial}
        \item{Modo de juego remoto (permitir que ambos jugadores
        utilizen una computadora diferente)}
        \item{Opcion para cambiar el tama\~no del tablero.}
        \item{Tutorial sobre como jugar}
\end{itemize}
{\bf Nota:} El estudiante puede proponer sus propios extras.

\end{document}