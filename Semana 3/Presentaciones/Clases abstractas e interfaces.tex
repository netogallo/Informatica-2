\documentclass{beamer}
\usepackage[utf8]{inputenc}
\usepackage{hyperref}
\usepackage{multicol}
\usepackage{hyperref}
\usepackage{verbatim}

\inputencoding{utf8}

\mode<presentation> {
    \usetheme{Madrid}
}

\usepackage{graphicx}
\usepackage{booktabs}

\title{Clases Abstractas e Interfaces}
\author{Ernesto Rodriguez}
\institute{
    Universidad del Itsmo \\
    \medskip \textit{erodriguez@unis.edu.gt}
}

\date[\today]{}

\begin{document}

\begin{frame}
\titlepage
\end{frame}

\begin{frame}
    \frametitle{Clases Abstractas: Motivaci\'on}
    \begin{itemize}
        \item{Muchas veces tenemos una famila de objetos que comparten caracteristicas en comun.}
        \item{Sin embargo, no tiene sentido o utilidad crear instancias de un objeto arriba en la jerarquia de clases.}
        \item{Esto podria resultar en una clase que nunca sera instanciada directamente.}
        \item{Muchas veces, esa clase va tener metodos vacios, con el proposito que las sub-clases los cambien.}
    \end{itemize}
    {\bf Soluci\'on:} \emph{Clases abstractas}
\end{frame}

\begin{frame}
    \frametitle{Clases Abstractas}
    \begin{itemize}
        \item{Clases cuyo proposito no es crear instancias, sino permitir
            que otras clases compartan propiedades y metodos.}
        \item{No es possible crear una instancia de una clase abstracta.}
        \item{Sin embargo, una clase abstracta puede utilizarse como tipo,
            ya sea de una propiedad o variable.}
        \item{La clase abstracta puede definir \emph{metodos abstractos}.}
        \item{Se declaran mediante la palabra reservada \texttt{abstract}.}
    \end{itemize}
\end{frame}

\begin{frame}
    \frametitle{Metodos Abstractos}
    \begin{itemize}
        \item{Un tipo especial de metodo que solo pede existir en una clase abstracta.}
        \item{El metodo no tiene cuerpo, sin embargo tiene una \emph{firma}.}
        \item{La \emph{clase abstracta} puede hacer uso de sus \emph{metodos abstractos}.
            Incluso, a pesar que el metodo no tiene cuerop.}
        \item{Toda clase ordinaria que herede de una clase abstracta {\bf debe} definir
            un cuerpo para todos los \emph{metodos abstractos}}.
        \item{Una clase abstracta puede heredar de otra clase abstracta {\bf sin}
            necesidad de implementar sus \emph{metodos abstractos}.}
        \item{Tambien se definen mediante la palabra reservada \texttt{abstract}.}
        \end{itemize}
\end{frame}

\begin{frame}
    \frametitle{Interfaces: Moticaci\'on}
    \begin{itemize}
        \item{En C\#, una clase solo puede heredar de otra clase (El problema del diamante).}
        \item{Sin embargo, existen ocasiones en que dos objetos completamente
            diferentes comparten algunas operaciones:
                \begin{itemize}
                    \item{Biblioteca y Bosque: \'ambos son collecciones.}
                    \item{Vehiculo y Animal: \'ambos tienen una posici\'on y pueden moverse}
                \end{itemize}
            }
        \item{A menudo, queremos escribir un m\'etodo que pueda utilizar ambos
            objetos, pero ¿que \emph{firma} tendria el metodo?}
    \end{itemize}
{\bf Soluci\'on: } Interfaces
\end{frame}

\begin{frame}
    \frametitle{Interfaces}
    \begin{itemize}
        \item{Definen un contrato, que debe debe cumplir toda clase
            que implemente la interfaz}
        \item{Son una lista de metodos y propiedads, sin definir cuerpo}
        \item{No se pueden crear instancias de interfaces, sin embargo,
            pueden utilizarse como tipo de propiedades y variables}
        \item{Se definen mediante la palabra reservada \texttt{interface}}
        \item{Una clase declara que implementa una interfaz mediante la
            palabra reservada \texttt{implements}}
    \end{itemize}
\end{frame}

\begin{frame}
    \frametitle{Interfaces: Aplicaciones}
    \begin{itemize}
        \item{Una herramienta que permite crear abstracciones poderosas.}
        \item{Permite escribir metodos que pueden utilizarse con una variedad de objetos.}
        \item{Permite crear bibliotecas, donde el programador simplemnte debe implementar
            una serie de interfaces para poder utilizar todas las funciones.}
        \item{Permiten que un objeto imite a otro objeto para propositos de
            evaluaci\'on o demostraci\'on.}
    \end{itemize}
\end{frame}
\end{document}