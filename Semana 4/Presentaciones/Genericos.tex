\documentclass{beamer}
\usepackage[utf8]{inputenc}
\usepackage{hyperref}
\usepackage{multicol}
\usepackage{hyperref}
\usepackage{verbatim}

\inputencoding{utf8}

\mode<presentation> {
    \usetheme{Madrid}
}

\usepackage{graphicx}
\usepackage{booktabs}

\title{Genericos}
\author{Ernesto Rodriguez}
\institute{
    Universidad del Itsmo \\
    \medskip \textit{erodriguez@unis.edu.gt}
}

\date[\today]{}

\begin{document}

\begin{frame}
\titlepage
\end{frame}

\begin{frame}
    \frametitle{Introducci\'on}
    \begin{itemize}
        \item{Los tipos genericos probablemente abarcan un 50\% de los
        estudios que he hecho en el campo de computaci\'on.}
        \item{Muchos lenguajes de programaci\'on se crean con el
        proposito de mejorar su manejo de tipos genericos.
        Ejemplo: Scala, Haskell, OCaml, Agda, ect.}
        \item{Existen cientos de articulos y publicaciones alrededor
        del concepto de tipos genericos.}
        \item{Para los que ya conozcan el tema, sientanse libres de
        aprender Haskell (\url{http://learnyouahaskell.com/}) para profundizar
        sus conocimientos sobre Genericos.}
    \end{itemize}
\end{frame}

\begin{frame}
    \frametitle{Motivaci\'on}
    \begin{itemize}
        \item{Existen ocasiones donde una clase se puede generalizar
        para varios tipos.}
        \item{Sin embargo, al definir las propiedades y metodos de
        una clase, uno debe fijar un tipo especifico.}
        \item{Esto puede llevar a repetici\'on de codigo, en
        especial cuando hay muchos niveles de herencia.}
    \end{itemize}
\end{frame}

\begin{frame}
    \frametitle{Motivaci\'on}
    \begin{itemize}
        \item{En algunos casos, las interfaces pueden ser una
        abstraccion suficiente para poder utilizar varios tipos.}
        \item{En el peor de los casos, una variable se puede
        declarar como tipo \texttt{Object} y permitir que se
        asigne cualquier valor.}
        \item{Sin embargo esto puede llevar a problemas.}
        \item{Ejemplo tradicional de genericos: Una colecci\'on}
    \end{itemize}
\end{frame}

\end{document}