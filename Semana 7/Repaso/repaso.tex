%%%%%%%%%%%%%%%%%%%%%%%%%%%%%%%%%%%%%%%%%
% Programming/Coding Assignment
% LaTeX Template
%
% This template has been downloaded from:
% http://www.latextemplates.com
%
% Original author:
% Ted Pavlic (http://www.tedpavlic.com)
%
% Note:
% The \lipsum[#] commands throughout this template generate dummy text
% to fill the template out. These commands should all be removed when 
% writing assignment content.
%
% This template uses a Perl script as an example snippet of code, most other
% languages are also usable. Configure them in the "CODE INCLUSION 
% CONFIGURATION" section.
%
%%%%%%%%%%%%%%%%%%%%%%%%%%%%%%%%%%%%%%%%%

%----------------------------------------------------------------------------------------
%	PACKAGES AND OTHER DOCUMENT CONFIGURATIONS
%----------------------------------------------------------------------------------------

\documentclass{article}

\usepackage{fancyhdr} % Required for custom headers
\usepackage{lastpage} % Required to determine the last page for the footer
\usepackage{extramarks} % Required for headers and footers
\usepackage[usenames,dvipsnames]{color} % Required for custom colors
\usepackage{graphicx} % Required to insert images
\usepackage{listings} % Required for insertion of code
\usepackage{courier} % Required for the courier font
\usepackage{multirow}
\usepackage{hyperref}


% Margins
\topmargin=-0.45in
\evensidemargin=0in
\oddsidemargin=0in
\textwidth=6.5in
\textheight=9.0in
\headsep=0.25in

\linespread{1.1} % Line spacing

%----------------------------------------------------------------------------------------
%	CODE INCLUSION CONFIGURATION
%----------------------------------------------------------------------------------------

\definecolor{MyDarkGreen}{rgb}{0.0,0.4,0.0} % This is the color used for comments
\lstloadlanguages{c} % Load Perl syntax for listings, for a list of other languages supported see: ftp://ftp.tex.ac.uk/tex-archive/macros/latex/contrib/listings/listings.pdf
\lstset{language=[sharp]c, % Use Perl in this example
        frame=single, % Single frame around code
        basicstyle=\small\ttfamily, % Use small true type font
        keywordstyle=[1]\color{Blue}\bf, % Perl functions bold and blue
        keywordstyle=[2]\color{Purple}, % Perl function arguments purple
        keywordstyle=[3]\color{Blue}\underbar, % Custom functions underlined and blue
        identifierstyle=, % Nothing special about identifiers                                         
        commentstyle=\usefont{T1}{pcr}{m}{sl}\color{MyDarkGreen}\small, % Comments small dark green courier font
        stringstyle=\color{Purple}, % Strings are purple
        showstringspaces=false, % Don't put marks in string spaces
        tabsize=5, % 5 spaces per tab
        %
        % Put standard Perl functions not included in the default language here
        morekeywords={rand},
        %
        % Put Perl function parameters here
        morekeywords=[2]{on, off, interp},
        %
        % Put user defined functions here
        morekeywords=[3]{test},
       	%
        morecomment=[l][\color{Blue}]{...}, % Line continuation (...) like blue comment
        numbers=left, % Line numbers on left
        firstnumber=1, % Line numbers start with line 1
        numberstyle=\tiny\color{Blue}, % Line numbers are blue and small
        stepnumber=5 % Line numbers go in steps of 5
}

\newcommand{\horrule}[1]{\rule{\linewidth}{#1}}

% Creates a new command to include a perl script, the first parameter is the filename of the script (without .pl), the second parameter is the caption
\newcommand{\perlscript}[2]{
\begin{itemize}
\item[]\lstinputlisting[caption=#2,label=#1]{#1.cs}
\end{itemize}
}

\begin{document}

\begin{tabular}{l l}
\multirow{5}{*}{\includegraphics[width=2cm]{../../Recursos/logo.png}} & Universidad del Istmo de Guatemala \\
 & Facultad de Ingenieria \\
 & Ing. en Sistemas \\
 & Informatica 2 \\
 & Prof. Ernesto Rodriguez - \href{mailto:erodriguez@unis.edu.gt}{erodriguez@unis.edu.gt} \\
\end{tabular}
\\\\\\

\begin{center}
        \horrule{0.5pt}
        \huge{Repaso Parcial \#2} \\
        \horrule{1pt}
\end{center}

Los \emph{algoritmos geneticos} son algoritmos de busqueda que permiten encontrar soluciones
a problemas que son dificiles de resolver desde un punto de vista practico. Estos algoritmos
se inspiran en el mecanismo que utilizan los seres vivos para adaptarse a ambientes diferentes.

\section*{Interfaz IProblem}
Los problemas de busqueda utilizando programaci\'on genetica seran representados por
la interfaz generica $\mathtt{IProblem\langle T\rangle}$. {\bf Como primer ejercicio, 
debe implementar la interfaz \texttt{IProblem} la cual tendra los siguientes metodos:}
\\\\
El primer metodo es el \emph{operador de intercambio}. Dicho
operador lo representaremos con metodo llamado \texttt{Crossover} de tipo ``$\mathtt{Crossover}\ 
:\ \mathtt{T}[\ ]\otimes\mathtt{T}[\ ]\rightarrow\mathtt{T}[\ ]$''. Este operador recive
dos arreglos, los cuales son aproximaciones a una soluci\'on representadas como arreglos,
y las intenta combinar para obtener una mejor soluci\'on.
\\\\
El segundo elemento de un algoritmo genetico es el \emph{operador de mutaci\'on}. Este
operador lo representaremos con un metodo llamado \texttt{Mutate} de tipo ``$\mathtt{Mutate}\ :\ 
\mathtt{T}[\ ]\rightarrow\mathtt{T}[\ ]$''. El cual se le proporciona una aproximaci\'on
de soluci\'on, y la modifica de forma aleatorea para obtener una nueva possible soluci\'on.
\\\\
El tercer elemento de un algoritmo genetico es el \emph{criterio de rendimiento}. Este
criterio evalua que tan buena o mala es una soluci\'on en particular. Este criterio se
representara con un metodo llamado $\mathtt{Score}$ de tipo $\mathtt{Score}\ :\ \mathtt{T}[\ ]
\rightarrow \mathtt{double}$. 
\\\\
El ultimo elemento del algoritmo es un metodo que permite generar soluciones aleatoreas.
Este metodo llamado $\mathtt{Guess}$ tiene tipo $\mathtt{Guess}\ :\ \mathtt{void}\rightarrow
\mathtt{T}[\ ]$. Este metodo no recive ningun parametro y retorna una soluci\'on generada
al azar del problema en mano.
\subsection*{Problema del vendedor ambulante}
El problema del vendedor ambulante es uno de los problemas m\'as ``molestamente complejo''
que hay en la computaci\'on. Consiste en encontrar el camino m\'as corto que recorra un
grupo de ciudades y que luego regresa a la ciudad de partida. Este problema es un canditato
perfecto para aproximar con algoritmos geneticos. Supongamos que se desea recurrer una cantidad
de $n$ ciudades. Cada ciudad se representara con un numero $0\leq i < n$. Con esto, podemos
representar un reccorido $r$ como un arreglo de enteros $[i_0 \ldots i_n]$ en donde cada indice
del arreglo, representa la ciudad a la que se viajara luego de haber visitado el indicie anterior.
Supongamos que tenemos las ciudades:
\begin{enumerate}
        \item Guatemala
        \item Managua
        \item San Salvador
\end{enumerate}
Entonces, el arreglo $[2,1,3]$ representa un recorrido que empieza en Managua, luego
pasa por Guatemala, de ahi pasa por San Salvador para luego regresar a Managua. Las distancias
entre ciudades, se pueden representar con un arreglo de dos dimensiones llamado $d$ donde
una pareja de indices representa la distancia entre dos puntos. Por ejemplo en el ejemplo
anterior, las distancias se representarian con el arreglo de dos dimensions (matriz):
\\\\
\begin{tabular}{|c|c|c|}
        \hline
        0 & 1020 & 454\\
        \hline
        1020 & 0 & 511 \\
        \hline
        454 & 511 & 0 \\
        \hline
\end{tabular}
\\\\
La cual muestra la distancia entre las ciudades en kilometros por tierra. Para evaluar
que tan buena es una soluci\'on, simplemente se suman las distancias del recorrido. Para
el recorrido $[2,1,3]$, su puntaje seria $1020+454+511$ o $1985$.
\\\\
Con esta informaci\'on, se puede implementar la clase abstracta \texttt{TSP} (Traveling salesman problem), con los siguientes
criterios:
\begin{itemize}
        \item{La clase \texttt{TSP} debe implementar la interfaz $\mathtt{IProblem \langle int\rangle}$.}
        \item{El constructor de \texttt{TSP} debe recibir un arreglo de dos dimensiones (matriz) de \texttt{double}, el cuar representara
        las distancias entre un conjunto de ciudades, como se explico anteriormente.}
        \item{Los metodos \texttt{Crossover} y \texttt{Mutate} deben ser abstractos}
        \item{El metodo \texttt{Score} utiliza la matriz pasada en el construcotr para caluclar
        la distancia de un recorrido en particular y la retorna.}
        \item{El metodo \texttt{Guess} retorna un recorrido (array de enteros) al azar. Tomar
        en cuenta que cada numero {\bf solo debe aparecer una vez} en el arreglo.}
\end{itemize}
{\bf Implementar la clase abstracta TSP}

\section*{Instancia concreta de TSP}
Los metodos \texttt{Crossover} y \texttt{Mutate} para \texttt{TSP} tienen varias posibles
implementaci\'ones. El rendimiento del algoritmo genetico depende de dichas implementaciones.
En esta secci\'on consideraremos las siguientes:
\\\\
Para el metodo \texttt{Mutate}, dado un arreglo, este metodo seleccionara dos indices
aleatoreamente e intercambiara los valores. Por ejemplo, para el arreglo $[2,1,3$, si
se seleccionan los indices 0 y 3, el resultado seria $[3,1,2]$;
\\\\
El metodo \texttt{Crossover}, construye un nuevo arreglo, primero combinando los dos arreglos
en un arreglo con el doble de longitud que los arreglos originales y luego seleccionando
los numeros del nuevo arreglo en orden hasta tener todos los numeros. En un arreglo de longitud
original. Por ejemplo, dados los arreglos $[3,1,2]$ y $[2,1,3]$, primero creamos el arreglo
$[3,2,1,1,2,3]$ y luego seleccionamos la primera ocurrencia de cada numero en orden para
obtener el arreglo $[3,2,1]$.
\\\\
{\bf Crear una clase llamada TSPEjemplo, la cual hereda de la clase TSP e implementa los
metodos Mutate y Crossover como esta descrito en esta seccion.}
\section*{La clase Solver}
Crear una clase llamada \texttt{Solver}, la cual tiene un metodo estatico generico llamado
\texttt{Solve} de tipo $\mathtt{Solve\langle T\rangle}\ :\ \mathtt{IProblem\langle T\rangle}\otimes
\mathtt{double}\rightarrow \mathtt{T}[\ ]$. Este metodo busca una soluci\'on al problema que
se le dio, la cual debe tener un punteo (o score) menor al segundo parametro. Para ello sigue
los siguientes pasos:
\begin{enumerate}
        \item{Utilizar el metodo \texttt{Guess} para generar 5 soluciones aleatoreas}
        \item{Generar otras 5 soluciones aplicando el metodo \texttt{Mutate} a cada soluci\'on}
        \item{De las 10 soluciones, seleccionar 5 parejas aleatoreamente, utilizar estas 5 parejas
        para generar otras 5 soluciones m\'as utilizando el metodo \texttt{Crossover}}
        \item{De las 15 soluciones, seleccionar las 5 con puntaje o score m\'as bajo, descartar
        el resto.}
        \item{Si alguna de estas 5 soluciones tiene un puntaje menor al segundo parametro de
        este metodo, retornar dicha soluci\'on, de lo contrario, ir al paso 2 y repetir.}
\end{enumerate}
{\bf Implementar el algoritmo Solve.}
\section*{Optimizaciones}
Siempre se pueden considerar mejoras, por ejemplo:
\begin{itemize}
        \item{Cambiar el metodo corssover o mutate}
        \item{Utilizar un pool mas grande de soluciones candidatos}
\end{itemize}
Tienen la libertad de probar :)
\end{document}